\documentclass[]{article}

\usepackage{fullpage}
\usepackage{minted}
\usepackage{csquotes}
\usemintedstyle{friendly}

\title{COMP 479 - Project 4}
\author{François LaBerge}

\begin{document}
\maketitle

\section{Different Behaviours of Ranking Schemes}
\section{Issues with the tf ranked postings list}
No issues complicated issues were encountered while implementing the tf-ranked postings list. To keep the lists ranked during the indexing process we use python ``heapq'' library to create a priority queue that will keep our list sorted throughout the indexing process. The algorithm on page 45 of the 7th slide set was implemented to handle the ranking with a limited number of items. Here is a code snippet showing the implementation

\inputminted[]{python}{snip/tfranked.py}

Note that the negative signs are necessary to turn the min heap into a max heap. For querying, the sorted postings list is retrieved using heapq's ``nsmallest'' method. 

\inputminted[]{python}{snip/tfretrieve.py}

\section{Top 15 return functionality}
\section{Crawling}
We used two libraries to handle web crawling and scraping. We opted for ``Scrapy'' as our web crawling framework and ``Beatifulsoup'' as our tool for scraping. Scrapy was a good choice for webscraping as it is usually used to scrapy a single domain, which is our case in this project. 

\end{document}